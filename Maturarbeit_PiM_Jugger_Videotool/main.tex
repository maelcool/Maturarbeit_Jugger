\documentclass{article}
\usepackage{graphicx} % Required for inserting images
\usepackage[utf8]{inputenc}
\usepackage[german]{babel}
\usepackage{hyperref}
\usepackage{csquotes}
\usepackage[backend=biber, style=ieee]{biblatex}
\bibliography{references}

\title{Maturarbeit PiM Jugger Videotool}
\author{Mael Pittet}
\date{April 2025}

\begin{document}

\maketitle

\section{Einführung}
\section{Jugger}
\subsection{Was ist Jugger}
Jugger ist eine moderne Mannschaftssportart, die in den späten 1980er-Jahren vom postapokalyptischen Film The Salute of the Jugger (dt. Titel: Die Jugger – Kampf der Besten) inspiriert wurde. Seither hat sich der Sport weit vom Film entfernt und ein eigenständiges, strukturiertes Regelwerk entwickelt, das auf Fairness, Sicherheit und Dynamik ausgelegt ist. Innerhalb der Jugger-Community stellen die spanische und die deutsche Gruppe die zwei größten Gemeinschaften dar. Es lassen sich jedoch Unterschiede in verschiedenen Aspekten zwischen beiden Gruppen beobachten. Aufgrund der Tatsache, dass das einzige Schweizer Team "Jugger Basilisken Basel" in der deutschen Community aktiv ist, wird im vorliegenden Text lediglich auf das deutsche Regelwerk sowie auf die deutsche Spielpraxis eingegangen.
\subsection{Wichtige Jugger Begriffe}
\begin{itemize}
\item Jugg: ein zylindrischer Schaumstoff, welcher als Spielball verwendet wird 
\item Mal: Ein Donatförmiger gegenstand, in den der Jugg gesteckt werden muss, um einen Punkt zu ergattern
\item Pompfe: Die Spielgeräte, mit denen die Spieler*innen sich mit anderen Duellieren. Dabei gibt es:
\begin{itemize}
    \item Stab: hat eine Trefferflächer, sowie eine Blockfläche
    \item Langpompfe: hat nur eine Trefferfläche
    \item Q-Tipp: hat an den Enden jeweils eine Trefferfläche (Q-Tipp zu deutsch Wattestäbchen)
    \item Schild: eine Schild und dazu eine kürze Pompfe
    \item Doppelkurz: zwei kürzere Pompfen, welche mit jeweils einer Hand gehalten werden kann
    \item Kette: Eine 3.2m lange Schnur mit einem Schaumstoffball am Ende, gibt mehr Strafzeit
\end{itemize} 
\item Läufer*in: Person, welche den Jugg tragen darf, dafür allerdings keine Pompfe führen darf
\item Grün / Rot: Ganginger Begriff um während dem Spiel den eigenen Teammitgliedern mitzuteilen ob das eigene Team (grün) den Jugg Kontrolliert oder das Gegnerische (rot)
\item Druckpunkt: Eine Person, welche ihr Duell schnell ausspielen will, mit der Hoffnung bessere Chancen auf einen Sieg zu haben und somit dem gegnerischen Team in den Rücken fallen zu können
\item Steine: Die Messeinheit in Jugger, beträgt anderthalb Sekunden
\item Läufi / Spieli: Eine gängige Variante um alle Geschlechter anzusprechen, sowie die Wörter kurz zu halten, da während des Spiels lange Wörter zu lange brauchen.
\end{itemize}
\subsection{Spielablauf}
Um Jugger zu Spielen, braucht man 5 Personen. Davon muss genau eine Person Läufer*in sein und maximal eine Person darf eine Kette spielen. An Turnieren sind vier Schiedsrichter*innen gängig, wobei diese Zahl variieren kann. Einer der Schiedsrichtern beginnt das Spiel und zählt von 20 Runter im Takt der Steinen. Alle fünf Personen rennen zur Mitte, in der der Jugg liegt. Wer von einer Pompfe getroffen wird, muss für fünd Steine abknien und ist somit für siebeneinhalb Sekunden aus dem Spiel. Sollte eine Kette eine Person treffen, ist diese Person für acht Steine am knien (oder 12 Sekunden). Der Zug endet, sobald einer der beiden Läufer*innen den Jugg ins Mal Steck und somit dem Team den Punkt sichert.
\subsection{Wichtige Regeln}
\begin{itemize}
    \item Pinnen: Wenn eine Person kniet, darf eine andere Person kommen und ihre Pompfe auf die kniede Person legen. Diese Person darf nun nicht mehr aufstehen, bis die Person die Pompfe wegnimmt. Diesen Vorgang nennt man eine Person pinnen. \cite{JuggerRegelwerk}
\end{itemize}
\end{document}
